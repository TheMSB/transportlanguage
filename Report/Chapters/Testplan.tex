\chapter{Test plan}

The Open Transport Language Deluxe compiler consists of three parts.
We tested each part individually, while assuming that the other parts work as expected.
Additionally, we tested the whole compiler using a test program which features almost all possible operations.

We have used unit tests for the front end compiler and the intermediate representation.
The back end compiler does not feature many automated testing since there are no easy tools for automatically inspecting the resulting bytecode.
However, we wrote a unit test which compiles a program using all visitor methods on the back end compiler.
Furthermore, the ASM class writer calls are checked by ASM for abnormalities while compiling this program.
We also ran and inspected the resulting class file manually with \code{javap} and the IntelliJ decompiler.

\section{Test programs}

The compiler has been tested with a number of test programs. The first program is a program which is correct and uses (almost) all features of the language. This program is included in appendix \ref{chap:testprogram} on page \pageref{chap:testprogram}.

All other test programs are located in \code{Code/src/test/resources/otld/otld/} and are discussed below.

\begin{itemize}
\item
	\code{test.tldr} \\
	Description including compiler output
\end{itemize}
