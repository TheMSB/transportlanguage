\documentclass[10pt,a4paper,notitlepage]{article}
\usepackage[utf8]{inputenc}
\usepackage{graphicx}
\usepackage{listings}
\usepackage[T1]{fontenc}
\usepackage{lmodern}

\lstset{basicstyle=\ttfamily,frame=lines,columns=fullflexible}

\newcommand{\name}{Open Transport Language Deluxe }
\newcommand{\shortname}{\textsc{otld} }

\title{\name \\ Language specification}
\author{Martijn Bruning \\ Jan-Jelle Kester}

\begin{document}
\maketitle

\section{Layout of a program}

Defining a program in \shortname is like defining a program in Pascal, only in \shortname you are defining a \emph{city}. A city can be defined with the command \texttt{City <name>;}, where \texttt{<name>} is replaced with the name of the program.

\subsection{City blocks}

A city consists of a number of blocks, each containing integral parts of the program. Each block starts with the statement \texttt{Begin <block>;} and ends with \texttt{End <block>;}.

\begin{itemize}
\item \texttt{industry}: This block should contain all factories of the city.
\item \texttt{depot}: This block should contain all wagons and trains.
\item \texttt{track}: This block should contain all signals and waypoints.
\item \texttt{company}: This block should contain the program itself.
\end{itemize}

\subsubsection*{Example}

\begin{lstlisting}
Begin depot;
	Wagon w accepts coal;
	Train t accepts 6 steel;
End depot;
\end{lstlisting}

\section{Basic expressions}

\subsection{Constants}

\begin{itemize}
\item Integer: \verb=-?(0|[1-9][0-9]*)=
\item Boolean: \verb=red|green=
\item Character: \verb='[a-zA-Z0-9]'=
\end{itemize}

\subsection{Variable declarations}

Variables in \shortname are called \emph{wagons}. Wagons are statically typed. The type should be set on declaration.
A wagon can be declared with the statement \texttt{Wagon <id> accepts <type>;} where \texttt{<id>} is a unique identifier for the wagon and \texttt{<type>} is the type of the wagon.

Booleans are treated differently. Booleans are stored in a \emph{signal}. The declaration and assignment of signals is different from wagons. A signal can be declared with the statement \texttt{Signal <id> is <color>;} where \texttt{<color>} is either \texttt{red} or \texttt{green}.

\subsubsection*{Example}

\begin{lstlisting}
Wagon w1 accepts coal;
Wagon w2 accepts steel;
Signal s is green;
\end{lstlisting}

\subsection{Assignment}

A constant value or a value from another variable can be assigned to a variable. To assign a constant to a variable use the statement \texttt{Load <const> into wagon <id>;} where \texttt{<const>} is the constant and \texttt{<id>} is the identifier for the wagon. To copy the value of an existing wagon into another wagon use \texttt{Transfer wagon <id> to wagon <id>;}. The term `transfer' does not mean that the source wagon will be empty -- it remains untouched.

\subsubsection*{Example}

\begin{lstlisting}
Load 10 into wagon w1;
Transfer wagon w1 into wagon w2;
\end{lstlisting}

\subsection{Arithmetic operations}

There are two unary operations with each a special syntax:

\begin{itemize}
\item Unary minus: Inverts a wagon. \\ \texttt{Turn wagon <id> around;}
\item Not (for signals): Inverts a signal. \\ \texttt{Switch signal <id>;}
\end{itemize}

The other operations use the statement \texttt{Transport <id>, <id> to factory <op> and fully load <id>;}. The first to occurances of \texttt{<id>} are the left- and righthand variables for the operation, the third \texttt{<id>} is the destination wagon. The operation \texttt{<op>} can be any of the following:

\begin{itemize}
\item Addition: \texttt{add}
\item Subtraction: \texttt{subtract}
\item Multiplication: \texttt{multiply}
\item Division: \texttt{divide}
\item Modulo division: \texttt{modulo}
\item Less than: \texttt{complt}
\item Greater than: \texttt{compgt}
\item Less than or equal: \texttt{complte}
\item Greater than or equal: \texttt{compgte}
\item Equal: \texttt{compeq}
\item Logical and: \texttt{and}
\item Logical or: \texttt{or}
\end{itemize}

The operations that return a boolean can be used in two ways; either as mentioned above or with the statement \texttt{Transport <id>, <id> to factory <op> and set signal <id>;} to store the boolean in a signal.

\subsubsection*{Example}

\begin{lstlisting}
Signal s1 is green;
Transport w1, w2 to factory multiply and fully load w1;
Transport w1, w2 to factory complt and set signal s2;
Transport s1, s2 to factory and and set signal s1;
\end{lstlisting}

\subsection{Conditional statements}

Conditional statements are based on signals. A conditional statement starts with \texttt{Approach signal <id>;} where \texttt{<id>} is the signal that indicates which path to choose. Code under \texttt{Case green;} is executed when the signal is green, code under \texttt{Case red;} is executed when the signal is red. The statement ends with \texttt{Pass signal;}.

The \texttt{Case} statements are not required, these just function as the start points for each block. The code under these statements is executed up until another \texttt{Case} statement or \texttt{Pass signal}.

\subsubsection*{Example}

\begin{lstlisting}
Approach signal s;
	Case red:
		Write "Train delayed" to journal;
Pass signal;
\end{lstlisting}

\subsection{Loops}

In \shortname \emph{circles} can be used for loops. Before defining the loop itself, the loop condition (\emph{waypoint}) needs to be defined. A waypoint can be defined with \texttt{Waypoint <id>;}. A waypoint behaves exactly the same way as a signal, but with one difference. It is possible to programmatically define the value of the waypoint every time it is accessed. To do this, add a waypoint block directly after the waypoint (\texttt{Begin waypoint;} and \texttt{End waypoint;}). Code in this waypoint block should write a boolean value to the identifier of the waypoint.

The code that should be executed in each iteration should be in a circle block starting with \texttt{Begin circle <id>;} and ending with \texttt{End circle;} where \texttt{<id>} is the identifier of the waypoint. To break out of a circle the statement \texttt{Stop;} can be used.

\begin{lstlisting}
Waypoint p;
Begin waypoint;
	Transport i, length to factory complt and set signal p;
End waypoint;

Wagon i accepts mail;
Wagon length accepts mail;
Load 10 into wagon length;

Begin circle p;
	Write "This is executed a single time" to journal;
	Stop;
	Write "This is not executed" to journal;
End circle;
\end{lstlisting}

\subsection{Procedures and functions}

\subsection{Arrays}

\subsection{Records}

\end{document}
